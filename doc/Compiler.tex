%%%%%%%%%%%%%%%%%%%%%%%%%%%%%%%%%%%%%%%%%%%%%%%%%%%%%%%%%%%%%%%%%%%%%%%%%%%%%%%%
% Template for USENIX papers.
%
% History:
%
% - TEMPLATE for Usenix papers, specifically to meet requirements of
%   USENIX '05. originally a template for producing IEEE-format
%   articles using LaTeX. written by Matthew Ward, CS Department,
%   Worcester Polytechnic Institute. adapted by David Beazley for his
%   excellent SWIG paper in Proceedings, Tcl 96. turned into a
%   smartass generic template by De Clarke, with thanks to both the
%   above pioneers. Use at your own risk. Complaints to /dev/null.
%   Make it two column with no page numbering, default is 10 point.
%
% - Munged by Fred Douglis <douglis@research.att.com> 10/97 to
%   separate the .sty file from the LaTeX source template, so that
%   people can more easily include the .sty file into an existing
%   document. Also changed to more closely follow the style guidelines
%   as represented by the Word sample file.
%
% - Note that since 2010, USENIX does not require endnotes. If you
%   want foot of page notes, don't include the endnotes package in the
%   usepackage command, below.
% - This version uses the latex2e styles, not the very ancient 2.09
%   stuff.
%
% - Updated July 2018: Text block size changed from 6.5" to 7"
%
% - Updated Dec 2018 for ATC'19:
%
%   * Revised text to pass HotCRP's auto-formatting check, with
%     hotcrp.settings.submission_form.body_font_size=10pt, and
%     hotcrp.settings.submission_form.line_height=12pt
%
%   * Switched from \endnote-s to \footnote-s to match Usenix's policy.
%
%   * \section* => \begin{abstract} ... \end{abstract}
%
%   * Make template self-contained in terms of bibtex entires, to allow
%     this file to be compiled. (And changing refs style to 'plain'.)
%
%   * Make template self-contained in terms of figures, to
%     allow this file to be compiled. 
%
%   * Added packages for hyperref, embedding fonts, and improving
%     appearance.
%   
%   * Removed outdated text.
%
%%%%%%%%%%%%%%%%%%%%%%%%%%%%%%%%%%%%%%%%%%%%%%%%%%%%%%%%%%%%%%%%%%%%%%%%%%%%%%%%

\documentclass[lettersize,journal]{IEEEtran}
% \usepackage{usenix2019_v3}

% to be able to draw some self-contained figs
\usepackage{tikz}
\usepackage{amsmath}

% inlined bib file
\usepackage{filecontents}
\usepackage{cite}
\usepackage{amsmath,amssymb,amsfonts}
% \usepackage{algorithmic}
\usepackage{graphicx}
\usepackage{textcomp}
\usepackage{tabularx}
\usepackage{subcaption}
\usepackage{makecell}
\usepackage{url}
\usepackage[switch]{lineno}
% \linenumbers
\usepackage{listings}
\usepackage{multirow}
% \usepackage[xcdraw]{xcolor}
% \usepackage{graphicx}
% \usepackage{textcomp}
% \usepackage{tabularx}
% \usepackage{adjustbox}

\captionsetup{compatibility=false}
\usepackage[margin=0.5in]{geometry}
\usepackage{textcomp}

% \usepackage{bera}% optional: just to have a nice mono-spaced font
\usepackage{listings}
% \usepackage{xcolor}

\colorlet{punct}{red!60!black}
\definecolor{background}{HTML}{EEEEEE}
\definecolor{delim}{RGB}{20,105,176}
\colorlet{numb}{magenta!60!black}

% %-------------------------------------------------------------------------------
% \begin{filecontents}{\jobname.bib}
% %-------------------------------------------------------------------------------
% @Book{arpachiDusseau18:osbook,
%   author =       {Arpaci-Dusseau, Remzi H. and Arpaci-Dusseau Andrea C.},
%   title =        {Operating Systems: Three Easy Pieces},
%   publisher =    {Arpaci-Dusseau Books, LLC},
%   year =         2015,
%   edition =      {1.00},
%   note =         {\url{http://pages.cs.wisc.edu/~remzi/OSTEP/}}
% }
% @InProceedings{waldspurger02,
%   author =       {Waldspurger, Carl A.},
%   title =        {Memory resource management in {VMware ESX} server},
%   booktitle =    {USENIX Symposium on Operating System Design and
%                   Implementation (OSDI)},
%   year =         2002,
%   pages =        {181--194},
%   note =         {\url{https://www.usenix.org/legacy/event/osdi02/tech/waldspurger/waldspurger.pdf}}}
% \end{filecontents}


\begin{document}

\title{ An Open-Source P4\textsubscript{16} Compiler Backend for Reconfigurable Match-Action Table Switches:Making Networking Innovation Accessible}

\author{Debobroto Das Robin,  Javed I. Khan \\
Kent State University}
        % <-this % stops a space
% \thanks{This paper was produced by the IEEE Publication Technology Group. They are in Piscataway, NJ.}% <-this % stops a space
% \thanks{Manuscript received April 19, 2021; revised August 16, 2021.}}

% The paper headers
\markboth{Journal of \LaTeX\ Class Files,~Vol.~14, No.~8, August~2021}%
{Shell \MakeLowercase{\textit{et al.}}: A Sample Article Using IEEEtran.cls for IEEE Journals}

% \IEEEpubid{0000--0000/00\$00.00~\copyright~2021 IEEE}
% Remember, if you use this you must call \IEEEpubidadjcol in the second
% column for its text to clear the IEEEpubid mark.

\maketitle

%-------------------------------------------------------------------------------
\begin{abstract}
  %-------------------------------------------------------------------------------
  
TODO: Write your abstract here

  
\end{abstract}

\begin{IEEEkeywords}
  Programmable data plane, Compiler, Programmable switch,  Software-Defined Networking, 
   P4,   Open-source
  \end{IEEEkeywords}

\maketitle


%-------------------------------------------------------------------------------
\section{Introduction}
%-------------------------------------------------------------------------------
TODO write intro section here. You can add as many section you want with \\section command. 



\subsection{Sub section 1}\label{Subsection1}

You can add sub section here. 


\subsection{Sub section 2}\label{Subsection2}

You can add sub section 2 here. Or you can add more subsection under any section. I am cross 
refferencing the sub section here ~\ref{Subsection1}. You just need to add a \textbf{label} on any section or subsection you want 
refer at any point. Then you need to add a command similar to this \~\\ref{Subsection1}. 
You can also bold any text with \\textbf\{\} and italic with \\textit\{\}  command. 
Here is an 

\subsection{Subsection with table}

You can add table also. there are many nice latex table builder online. They will give you option to edit the data in the table manualy 
And generate the latex code for you. Then you can copy the latex code here. Or you can build the table in power point and 
export  it a s a figure. Then add the figure in the latex file.

Following is a sample table with latex code. the command for adding figure is given at the later part of the file. 


\begin{table}[h]
  \centering{
  \begin{tabular}{|l|l|}
  \hline
  Column 1 & Column 2 \\ \hline
  data     & data     \\ \hline
  data     & data     \\ \hline
  \end{tabular}
  }
\end{table}





%========================================================================================================================
\section{Background} \label{BackgroundSection} 
%========================================================================================================================


Here is an example of how to add a figure. You can add any figure with this command. 




\begin{figure}[h]
 \centering
 \includegraphics[trim=0in 1in 0in 0, clip,scale=.345]{Pipeline.pdf}
 % \caption{\centering V1Model pipeline architecture}
 \caption{ V1Model pipeline architecture}
 \label{fig:V1ModelArchitecture}
\end{figure}












%========================================================================================================================
\section{Conclusion}  \label{Conclusion}
%========================================================================================================================
Write your conclusin here. There are some reference to my papers in the compiler.bib file. You can cite them in your report. You can
also add citation for any other paper or report or website. The command for citing some work is 
~\cite{robin2022clb}, ~\cite{robin2021p4kp}, ~\cite{robin2022p4te}, ~\cite{robin2022preprint}. But before adding citation you need to add the bibtex entry in the compiler.bib file. 


%========================================================================================================================
\section{How to compile} 
%========================================================================================================================
There is a Makefile in the folder. It lists the steps to compile the latex file and generate the pdf. Alternatively you can take the 
compiler.tex and compiler.bib file and any other figure you want to use; then use the overleaf online editor. That is easier because it will not need
installing latex in your own PC. Easy writing. 

%-------------------------------------------------------------------------------
% Don't change the follwoing two lines. They are for formatting the file. 
\bibliographystyle{unsrt}
\bibliography{Compiler}



%%%%%%%%%%%%%%%%%%%%%%%%%%%%%%%%%%%%%%%%%%%%%%%%%%%%%%%%%%%%%%%%%%%%%%%%%%%%%%%%

\appendix
\subsection{\textit{QoS-Modifer} P4 Program}\label{App:QoSModiferP4Program}
%==========================================================================
This is an example if you want to write an algorithm or add source code. 

\small{
\begin{lstlisting}[linewidth=\columnwidth,breaklines=true,frame = single]
{
header control_packet_t {
bit<8> ipv4_diffserv;
bit<8> ipv6_trafficClass;
bit<8> index;
}
header ipv6_t {
bit<4>   version;
bit<8>   trafficClass;
bit<20>  flowLabel;
bit<16>  payloadLen;
bit<8>   nextHdr;
bit<8>   hopLimit;
bit<128> srcAddr;
bit<128> dstAddr;
}

}
\end{lstlisting}
}











\end{document}

%%%%%%%%%%%%%%%%%%%%%%%%%%%%%%%%%%%%%%%%%%%%%%%%%%%%%%%%%%%%%%%%%%%%%%%%%%%%%%%%

%%  LocalWords:  endnotes includegraphics fread ptr nobj noindent
%%  LocalWords:  pdflatex acks